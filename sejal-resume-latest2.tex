%%%%%%%%%%%%%%%%%%%%%%%%%%%%%%%%%%%%%%%
% Deedy CV/Resume
% XeLaTeX Template
% Version 1.0 (5/5/2014)
%
% This template has been downloaded from:
% http://www.LaTeXTemplates.com
%
% Original author:
% Debarghya Das (http://www.debarghyadas.com)
% With extensive modifications by:
% Vel (vel@latextemplates.com)
%
% License:
% CC BY-NC-SA 3.0 (http://creativecommons.org/licenses/by-nc-sa/3.0/)
%
% Important notes:
% This template needs to be compiled with XeLaTeX.
%
%%%%%%%%%%%%%%%%%%%%%%%%%%%%%%%%%%%%%%

\documentclass[letterpaper]{deedy-resume} % Use US Letter paper, change to a4paper for A4 
\usepackage{epstopdf}
\begin{document}

%----------------------------------------------------------------------------------------
%	TITLE SECTION
%----------------------------------------------------------------------------------------

%\lastupdated % Print the Last Updated text at the top right

\namesection{Sejal}{Chauhan}{ % Your name

\href{mailto:sejalch@cs.wisc.edu}{sejalc@cs.wisc.edu} | (608)960.5705 \\% Your contact information
}

%----------------------------------------------------------------------------------------
%	LEFT COLUMN
%----------------------------------------------------------------------------------------

\begin{minipage}[t]{0.23\textwidth} % The left column takes up 33% of the text width of the page

%------------------------------------------------
% Links
%------------------------------------------------

\section{Connect @}
\footnotesize{
\includegraphics[height=10pt]{Linkedin-256.png} LinkedIn:  \href{https://www.linkedin.com/in/sejalchauhan}{sejalchauhan} \\
\includegraphics[height=10pt]{Facebook-256.png} Facebook:  \href{https://www.facebook.com/sejal.chauhan1}{sejal.chauhan1} \\
\includegraphics[height=10pt]{GitHub-Mark-32px.png} GitHub:  \href{https://github.com/SejalChauhan}{SejalChauhan} \\
}
%\vspace{-6 mm} % Some whitespace after the section

%------------------------------------------------
% Skills
%------------------------------------------------

\section{Skills}
\subsection{Programming}
\footnotesize{C \textbullet{} C++ \textbullet{} Matlab \textbullet{}
Python \\ R \textbullet{} MySQL \textbullet{} Groovy\\}
\vspace{-1 mm}
\subsection{Tools}
Simulink \textbullet{} PSpice \textbullet{}
 Altera Quartus \\ Xilinx SDK \textbullet{}
 ModelSim \textbullet{} Verilog\\
R studio \textbullet{} MyEclipse\\
\vspace{-1 mm}
\subsection{Hardware}
Altera Cyclone II EP2C35 \\
Xilinx ML605 \textbullet{} Raspberry Pi\\
\vspace{-1 mm}
\subsection{Others}
802.11 \textbullet{} RTOS \textbullet{} Linux \\
ARMv8 Architecture and Design \textbullet{}\\
Embedded Systems\\
%\vspace{-5 mm} % Some whitespace after the section

%------------------------------------------------
% Coursework
%------------------------------------------------

\section{Coursework}
\subsection{Graduate}
\textbullet{} Machine Learning \\
\textbullet{} Operating Systems \\
\textbullet{} Algorithm Design \\

\subsection{Undergraduate}
\textbullet{} Signal Transformations \\
\textbullet{} Network Analysis \\
\textbullet{} Data Structures \\
\textbullet{} Probability Theory and \\ \hphantom{\textbullet{}} Stocastic Processes \\
\textbullet{} Communication Theory \\
\textbullet{} Computer Networks \\
\textbullet{} Microprocessor Systems \\
\textbullet{} Cellular and Mobile \\ \hphantom{\textbullet{}} Communications \\
\textbullet{} Object Oriented Programming\\

\subsection{Qualcomm}
\textbullet{} ARMv8 Architecture and Design \\

\subsection{Projects}
\footnotesize{\textbullet{} Machine Learning to identify \\ \hphantom{\textbullet{}} Habitable Exoplanets}\\
\footnotesize{\textbullet{} Multi-threaded Web Server}\\
\footnotesize{\textbullet{} Kernel thread support for xv6 OS}\\
\footnotesize{\textbullet{} Simple Shell} \\
\footnotesize{\textbullet{} MLFQ Scheduler for xv6 OS} \\

\hfill
\end{minipage} % The end of the left column
%
%----------------------------------------------------------------------------------------
%	RIGHT COLUMN
%----------------------------------------------------------------------------------------
%
\begin{minipage}[t]{0.76\textwidth} % The right column takes up 66% of the text width of the page


%------------------------------------------------
% Education
%------------------------------------------------

\section{Education}

\subsection{University of Wisconsin - Madison}
\descript{MS Computer Science}
\location{August 2015 - Till date | Madison, Wisconsin}
\vspace{\topsep} % Hacky fix for awkward extra vertical space
\begin{tightitemize}
\item Working on implementing a Bluetooth systems module on Raspberry Pi 2 with snappy ubunutu for the startup Paradrop(Exis). The aim is to develop a Bluetooth framework
 which the developers can use to make Bluetooth enabled Apps for the \textit{Smart Router}
\end{tightitemize}

\sectionspace % Some whitespace after the section
\vspace{1 mm}
\subsection{National Institute of Technology, Warangal}
\descript{B.Tech Electronics and Communication Engineering}
\location{August 2008 – May 2012 | Warangal, India}
\location{Cum. GPA 7.85/10}
\begin{tightitemize}
\item Concentrated on projects and internships in the areas of Embedded systems' design and development.
\end{tightitemize}

%------------------------------------------------
% Experience
%------------------------------------------------

\section{Experience}

\runsubsection{Epistemic Games}
\descript{| Graduate Research Assistant}
\location{August 2015 – Till Date | Wisconsin, Madison}
\begin{tightitemize}
\item Working with Epistemic Games in Wisconsin Center for Educational Research (WCER) to enable students to simulate internship experience via online games. The work involves maintaining the \textit{autoencoder}.
\item Also involved in maintaining the framework which fetches the chat data using Groovy from MySQL Database.
\item Work on porting some of the Groovy code to R for better performance.
\end{tightitemize}

\sectionspace % Some whitespace after the section
%\vspace{-1 mm} 

\runsubsection{Qualcomm}
\descript{| Engineer}
\location{July 2012 – July 2015 | Hyderabad, India}
\vspace{-1 mm} % Hacky fix for awkward extra vertical space
\begin{tightitemize}
\item Involved in design and development of firmware and Linux device drivers for Qualcomm's wireless chipsets.
\item Primary functional area of work involved developing 802.11r Scanning and implementation of Android’s Preferred Network Offload support in firmware with Privacy feature in Lollipop which is the latest Android version and has been pushed in millions of chipsets.
\item Worked on various memory optimizations and power save mechanisms that enabled faster connectivity. This led to a lesser die size and longer battery life with prolonged connectivity.
\item Used Lauterbach TRACE32 to analyze system stability issues and was the owner of wlan firmware stability. Wrote scripts to analyze the memory dump for faster triaging.
\item Developed insight in the new wireless protocols like WFA's - WPS, P2P, WMM, NaN and IEEE 802.11ac/p/ah that influence the device to device communication.
\item As an active member of Qualcomm Women in Science and Engineering (QWISE), organized various events including motivating talks by influential women in the field of Computer Science.
\end{tightitemize}

%\vspace{-1 mm} % Some whitespace after the section

%------------------------------------------------

\sectionspace % Some whitespace after the section
\runsubsection{Qualcomm}
\descript{| Industrial Internee}

\location{May 2011 – July 2011 | Hyderabad, India}
\begin{tightitemize}
\item Developed an understanding of 802.11 MAC implementation in the wireless device driver on both station and Soft Access Point. 
\item Worked on fixing Klocwork issues involving memory leaks, deallocation, dereferencing of null pointers and uninitialized variables that reduced the release cycle time, resource utilization and possible customer issues.
\item Selected as college campus ambassador for Qualcomm.
\end{tightitemize}

\vspace{-1 mm} % Some whitespace after the section

\end{minipage} % The end of the right column

%----------------------------------------------------------------------------------------
%	SECOND PAGE
%----------------------------------------------------------------------------------------
%----------------------------------------------------------------------------------------
%	LEFT COLUMN
%----------------------------------------------------------------------------------------

\begin{minipage}[t]{0.23\textwidth} % The left column takes up 33% of the text width of the page

\section{Awards \& Recognition}
\footnotesize{
\textbullet[2009]{} 2\textsuperscript{nd} in Litmux, Lantern and \\ \hphantom{\textbullet{}} Mock Press which tests oratory,
\\ \hphantom{\textbullet{}} language and communication skills.\\
\textbullet[2006]{} Awarded National \\ \hphantom{\textbullet{}} Scholarship by the Governor \\ \hphantom{\textbullet{}} of West Bengal for
Exceptional \\ \hphantom{\textbullet{}} performance in class X CBSE \\ \hphantom{\textbullet{}} Board Examinations.\\
\textbullet[2003]{} Placed 3\textsuperscript{rd} in shai \\ \hphantom{\textbullet{}} (sparring) in the International \\ \hphantom{\textbullet{}} South Asian
Goju Ryu Karate \\ \hphantom{\textbullet{}} Championship.\\
}

\section{Activities}
\footnotesize{
\textbullet{} Additional Secretary of the \\ \hphantom{\textbullet{}} ECE association - image processing \\ \hphantom{\textbullet{}} workshops to promote \\ \hphantom{\textbullet{}} student skills.\\
\textbullet{} Vice President of SEDS (Students \\ \hphantom{\textbullet{}} for Exploration and Development \\ \hphantom{\textbullet{}} of Space) NIT, Warangal which is\\ \hphantom{\textbullet{}} a college chapter of the\\ \hphantom{\textbullet{}} international organization with \\ \hphantom{\textbullet{}} NASA as its student advisor.\\
\textbullet{} Sub Core member of Technozion, \\ \hphantom{\textbullet{}} 2010 which takes care of the \\ \hphantom{\textbullet{}} overall conduction of the technical \\ \hphantom{\textbullet{}} fest’s events’ conduction.\\
\textbullet{} Built Gliders, CanSat (satellite in a \\ \hphantom{\textbullet{}} can) and radio in an inter collegiate \\ \hphantom{\textbullet{}} technical fest organized by NITW.\\
\textbullet{} Summited three mountains: \\ \hphantom{\textbullet{}} Mt Pangarchuliya(17,105 ft), \\ \hphantom{\textbullet{}} Mt Bhanoti(18,515 ft) and \\ \hphantom{\textbullet{}} Mt Shitidhar(16,214 ft) in The \\ \hphantom{\textbullet{}} Himalayas.\\
}
%------------------------------------------------
\hfill
\end{minipage} % The end of the left column
%
%
%----------------------------------------------------------------------------------------
%	RIGHT COLUMN
%----------------------------------------------------------------------------------------
%
\begin{minipage}[t]{0.76\textwidth} % The right column takes up 66% of the text width of the page


\runsubsection{Indian Institute of Technology, Bombay}
\descript{| Summer Intern}

\location{May 2010 – July 2010 | Mumbai, India}
\begin{tightitemize}
\item Development and Testing of Algorithms for Image and Video Compositing under Varying Illumination on Matlab.
\item Worked on self-illumination of the moving objects in a video with the help of masks obtained by Stauffer Grimson and Chan Vese active contours’ algorithm.
\item Automatic gain control effects of the camera were also removed for lesser noise which helped in extracting the best mask for the moving object with the help of k-means clustering and kalman filters.
\end{tightitemize}

%\vspace{-1 mm} % Some whitespace after the section

%------------------------------------------------

\runsubsection{ComNet}
\descript{| Industrial Internee}

\location{November 2009 – December 2009 | Gurgaon, India}
\begin{tightitemize}
\item Worked under “Activation, Discovery, Reconcilliation of System” project which was an internationally funded project to discover network devices.
\item It involved the reduction of the gap between M6 (MetaSolv Solution is a next-generation inventory and order management platform that enables the strategic and cost-effective delivery of traditional and next-generation services over complex networks) and A5 which is the activation part of actual network.
\end{tightitemize}
\vspace{-5 mm}
\hfill

%------------------------------------------------
% Research
%------------------------------------------------
\section{Academic Projects \& Research}
\runsubsection{Trading Accuracy for Power with an Under-designed Multiplier Architecture}

\location{December 2011 – May 2012 | Warangal, India}
\begin{tightitemize}
\item Researched a novel multiplier architecture with tunable error characteristics, that leverages a modified inaccurate 2x2 multiplier as its building block.
\item Our research showed that inaccurate multipliers achieve an average power saving of 31.78\% − 45.4\% over corresponding accurate multiplier designs, for an average error of 1.39\%−3.32\%.
\end{tightitemize}
\sectionspace % Some whitespace after the section
%------------------------------------------------
\runsubsection{Design of Conformal Antennas using Neural Networks}

\location{July 2011 – May 2012 | Warangal, India}
\begin{tightitemize}
\item Used neural networks to predict the best possible dimensions of an antenna patch that can be used on a surface of a cylinder or a cone at a particular given resonant frequency. 
\item The neural networks were trained initially by the results obtained from FEKO software.
\item The predictions obtained from the neural networks were tested by designing and analyzing the antennas.
\end{tightitemize}

\sectionspace % Some whitespace after the section
%------------------------------------------------
\runsubsection{Tracking a Mobile Adhoc Node in the mixed network}

\location{July 2011 – November 2011 | Warangal, India}
\begin{tightitemize}
\item The project simulated a mixed network with four wifi nodes and ten csma nodes using ns3. The wifi nodes were mobile and moved in a predefined area. The mobility model predefined a course change trace source which was used to simulate the trace events. Various tools were used to analyze the trace files while plotting probability of Beacon Reception.
\end{tightitemize}

\sectionspace % Some whitespace after the section
%------------------------------------------------
\runsubsection{Implementation of Rank Order Filter to Improve Image Quality}

\location{December 2010 – April 2011 | Warangal, India}
\begin{tightitemize}
\item Implemented software emulation for 2-D Rank Order Filter to remove specks while preserving the edges on Cyclone II EP2C35 FPGA.
\item To optimize memory image was first converted to a text file that was fed in SDRAM FIFO.
\item Implemented by adapting the bit serial approach by pipelining and parallel computing and that helped reduce the total CPU time considerably.
\end{tightitemize}

\sectionspace % Some whitespace after the section
%------------------------------------------------
\runsubsection{Digital Respiration Rate Meter}

\location{December 2010 – April 2011 | Warangal, India}
\begin{tightitemize}
\item Designed a circuit that counted the number of inhaling and exhaling cycles in one minute.
\item The system used a displacement transducer for sensing the respiration rate using IR transmitter and receiver.
\item This movement is sensed with the help of IR transmitter-receiver assembly of the sensing circuit and was converted into pulses through pulse generator. 
\item The respiration rate was displayed on a 3-digit display through the 7-segment decoder/driver.
\end{tightitemize}
\end{minipage} % The end of the right column
\end{document}
